\documentclass[11pt]{beamer}
\usetheme{Warsaw}
\usepackage[utf8]{inputenc}
\usepackage[english]{babel}
\usepackage{amsmath}
\usepackage{amsfonts}
\usepackage{amssymb}
\usepackage{graphicx}
\usepackage{pgfpages}
\author{Apaar Shanker \\ \texttt{apaar92@gmail.com}}
\title[Modelling Dendritic Solidification with Convection]{Influence of Convection on Microstructure Evolution during Solidification}
\setbeamercovered{transparent} 
\setbeamertemplate{navigation symbols}{} 
%\logo{} 
\institute{Indian Institute of Science} 
\date{} 
\subject{Bachelor's of Science (Research)} 
\begin{document}
	\begin{frame}
		\titlepage
	\end{frame}

	\begin{frame}
		\frametitle{Outline}
		\tableofcontents%[pausesections]
	\end{frame}
	
	\begin{frame}
		\begin{itemize}
			\item Solidification
			\item Importance of convection and need for incorporating it in model - defects like feckle formation
			\item Introduction to PF  -Idea of Functional, elaboration in bullets - eg - football, fracture etc. 
			\item Model Description - $\phi$ and $\mu$ evolution - highlight driving force and mass conservation
			\item one D and two D growth and Gibbs-Thompson
		\end{itemize}		
	\end{frame}
	\begin{frame}
		\begin{itemize}
			\item Incorporating Anisotropy by modifying functional
			\item Discretisation and implementation - Simulation Results
			\item Modifying Mass Conservation and incorporating Fluid Flow
			\item Implementation and Check Cases
			\item Conclusion and Future Plan
		\end{itemize}
	\end{frame}
	\section{Introduction}
	\begin{frame}
		\begin{itemize}
			\item Solidification is big chunk of material processing
			\item Micorstructure evolution very non uniform and process dependant
			\item Establish Process- Micro correlation
			\item 
		\end{itemize}		
	\end{frame}
	
	\section{Model Description}
	\begin{frame}
	\end{frame}
	
	\section{Results and Conclusion}
	\begin{frame}
	\end{frame}
	
	\begin{frame}
		\begin{center}
			Thank You
		\end{center}
	\end{frame}	
\end{document}


%\begin{frame}
%	\frametitle{What are Prime Numbers?}
%	\begin{definition}
%		A \alert{prime number} is a number that has exactly two divisors.
%	\end{definition}
%	\begin{example}
%		\begin{itemize}
%			\item 2 is prime (two divisors: 1 and 2).
%			\pause
%			\item 3 is prime (two divisors: 1 and 3).
%			\pause
%			\item 4 is not prime (\alert{three} divisors: 1, 2, and 4).
%		\end{itemize}
%	\end{example}
%\end{frame}
%
%\begin{frame}[t]
%	\frametitle{There Is No Largest Prime Number}
%	\framesubtitle{The proof uses \textit{reductio ad absurdum}.}
%	\begin{theorem}
%		There is no largest prime number.
%	\end{theorem}
%	\begin{proof}
%		\begin{enumerate}
%			\item<1-> Suppose $p$ were the largest prime number.
%			\item<2-> Let $q$ be the product of the first $p$ numbers.
%			\item<3-> Then $q + 1$ is not divisible by any of them.
%			\item<1-> Thus $q + 1$ is also prime and greater than $p$.\qedhere
%		\end{enumerate}
%	\end{proof}
%	\uncover<4->{The proof used \textit{reductio ad absurdum}.}
%\end{frame}
%
%\begin{frame}
%	\frametitle{What’s Still To Do?}
%	\begin{block}{Answered Questions}
%		How many primes are there?
%	\end{block}
%	\begin{block}{Open Questions}
%		Is every even number the sum of two primes?
%	\end{block}
%\end{frame}
