% $Header: /cvsroot/latex-beamer/latex-beamer/solutions/generic-talks/generic-ornate-15min-45min.en.tex,v 1.4 2004/10/07 20:53:08 tantau Exp $

\documentclass{beamer}

\mode<presentation>
{
  \usetheme{Warsaw}

  \setbeamercovered{transparent}
  % or whatever (possibly just delete it)
}


\usepackage[english]{babel}
% or whatever

\usepackage[latin1]{inputenc}
% or whatever

\usepackage{times}
\usepackage[T1]{fontenc}
% Or whatever. Note that the encoding and the font should match. If T1
% does not look nice, try deleting the line with the fontenc.


\title[] % (optional, use only with long paper titles)
{Quadrature Domains -- A Survey}

%\subtitle
%{Presentation Subtitle} % (optional)

\author[]% (optional, use only with lots of authors)
{Apaar Shanker \\ \texttt{} }
% - Use the \inst{?} command only if the authors have different
%   affiliation.

\institute[University of Alexandria] % (optional, but mostly needed)
{
 %\inst{1}%
  Indian Institute of Science
}
% - Use the \inst command only if there are several affiliations.
% - Keep it simple, no one is interested in your street address.

\subject{Talks}
% This is only inserted into the PDF information catalog. Can be left
% out.



% If you have a file called "university-logo-filename.xxx", where xxx
% is a graphic format that can be processed by latex or pdflatex,
% resp., then you can add a logo as follows:

% \pgfdeclareimage[height=0.5cm]{university-logo}{university-logo-filename}
% \logo{\pgfuseimage{university-logo}}



% Delete this, if you do not want the table of contents to pop up at
% the beginning of each subsection:
%\AtBeginSubsection[]
%{
%  \begin{frame}<beamer>
%    \frametitle{Outline}
%    \tableofcontents[currentsection,currentsubsection]
%  \end{frame}
%}


% If you wish to uncover everything in a step-wise fashion, uncomment
% the following command:

\beamerdefaultoverlayspecification{<+->}


\begin{document}

\begin{frame}
  \titlepage
\end{frame}

\begin{frame}
  \frametitle{Outline}

  \begin{itemize}

  \item {An exercise in Linear Algebra}

  \item {Quadrature Domains}

     \begin{itemize}
        \item {A definition and some remarks}
   
        \item {An example and some questions}

        \item {Relevance of the Cauchy transform}

     \end{itemize}

  \item {Properties of quadrature domains}

  \item {Constructing quadrature domains}

  \item {Examples}

  \item {What happens in higher dimensions?}

  \end{itemize}

  % \tableofcontents
  % You might wish to add the option [pausesections]
\end{frame}


% Since this a solution template for a generic talk, very little can
% be said about how it should be structured. However, the talk length
% of between 15min and 45min and the theme suggest that you stick to
% the following rules:

% - Exactly two or three sections (other than the summary).
% - At *most* three subsections per section.
% - Talk about 30s to 2min per frame. So there should be between about
%   15 and 30 frames, all told.

%%%%%%%%%%%%%%%%%%%%%%%%%%%%%%%%%%%%%%%%%%%%%%%%%%%%%%%%%%

\section{Introduction}

%\subsection[Short First Subsection Name]{First Subsection Name}

\begin{frame}
  \frametitle{An Exercise in Linear Algebra}
  % \framesubtitle{Subtitles are optional.}
  % - A title should summarize the slide in an understandable fashion
  %   for anyone how does not follow everything on the slide itself.

Let $V_n$ be the vector space of all real polynomials on $[a, b]$ with degree at most $n$. 
Choose distinct points $t_0, t_1, \ldots, t_n$ in $[a, b]$ and consider the following functionals
on $V_n$:
\[
L_i(p) = p(t_i), \; 0 \le i \le n.
\]

  \begin{itemize}
  \item Suppose that
\[
\alpha_0 L_0 + \alpha_1 L_1 + \ldots + \alpha_n L_n = 0.
\]
  \item Evaluate this linear combination on the basis elements $\{1, x, x^2, \ldots, x^n\}$ to get 
\[
\alpha_0 + \alpha_1 t_j + \ldots + \alpha_n t_j^n = 0
\]  
for all $1 \le j \le n$.

%Use \texttt{itemize} a lot.

  \end{itemize}
\end{frame}

%%%%%%%%%%%%%%%%%%%%%%%%%%%%%%%%%%%%%%%%%%%%%%%%%%%%%%%%%%%%%%%%%%%%%%%%%%%

\begin{frame}
  \frametitle{The Exercise Contd.}

 %You can create overlays\dots
  
\begin{itemize}
  \item These are $n + 1$ equations in the $n + 1$ unknowns $\alpha_0, \alpha_1, \ldots, \alpha_n$.
    
  \item Since the $t_i$'s were distinct, the only solutions are $\alpha_j = 0$ for all $ 1\le j \le n$.

  \item Hence there are constants $c_0, c_1, \ldots, c_n$ such that 
\[
\int_a^b p(x) \; dx = c_0 p(t_0) + c_1 p(t_1) + \ldots + c_n p(t_n)
\]
for all $p \in V_n$.

\end{itemize}

\end{frame}

%%%%%%%%%%%%%%%%%%%%%%%%%%%%%%%%%%%%%%%%%%%%%%%

\begin{frame}

\begin{itemize}

  \item Example: For all $p \in V_3$,
\[
\int_{-1}^1 f(t) \; dt = 1/3 f(-1) + 4/3 f(0) + 1/3 f(1)
\]
which is a `Quadrature Identity'. 

\medskip

  \item This motivates the following $\ldots$
 
\end{itemize}
\end{frame}

%%%%%%%%%%%%%%%%%%%%%%%%%%%%%%%%%%%%%%%%%%%%%%%%%%%%%%%%%%%%%%%%%%%%%%%%%%

\section{Quadrature Domains}

\begin{frame}
  \frametitle{A Definition and Remarks}

\begin{itemize}

 \item For a domain $D \subset \mathbb C$, let $L^1_{\mathcal O}(D) = L^1(D) \cap {\mathcal O}(D)$.

\medskip

 \item $D$ is a {\it quadrature domain} for $L^1_{\mathcal O}(D)$ if there exist points $z_1, z_2, \ldots, z_m \in D$ and scalars $a_{j, k} \in \mathbb C$ such that
\[
\int_D f(z) \; d  \sigma_z = \sum_{j = 1}^m \sum_{k = 0}^{r_j - 1} a_{j, k} f^{(k)}(z_j) 
\]
for every $f \in L^1_{\mathcal O}(D)$.

\medskip

 \item Quadrature data = The points $z_j$ (called the {\it nodes}) and the constants $a_{j, k}, r_j$.

% \item Can use other test classes too -- harmonic functions, subharmonic functions.

\end{itemize}

\end{frame}

%%%%%%%%%%%%%%%%%%%%%%%%%%%%%%%%%%%%%%%%%%%%%%%%%%%%%%%%%%%%%%%%%%%%%%%%%%%%%%

\begin{frame}
  \frametitle{An Example and Some Questions}

\begin{itemize}

  \item An Example: Take $D = B(a, r) \subset \mathbb C$. Then for all such $f$,
\[
\int_D f(z) \; d \sigma_z = \pi r^2 f(a).
\]

 \item Are there other examples of domains that admit such quadrature identities? How many are there?

\medskip

 \item What are the analytic properties of such domains?

\medskip

 \item If there are many such examples, is there a procedure to construct them given the quadrature data?

\end{itemize}

\end{frame}

%%%%%%%%%%%%%%%%%%%%%%%%%%%%%%%%%%%%%%%%%%%%%%%%%%%%%%%%%%%%%%%%%%%%%%%%%%%%%%%

\begin{frame}
 \frametitle{Relevance of the Cauchy transform}

\begin{itemize}
 
 \item Think of the quadrature identity as: The integral of $f$ on $D$ is a finite linear sum of Dirac masses and their derivatives at the nodes $z_j$.
 
 \item Take a quadrature domain $D \subset \mathbb C$. Pick $\zeta \in {\mathbb C} \setminus D$ and use 
\[
f(z) = 1/(z - \zeta)
\]
in the quadrature identity.

 \item Then
\[
\int_D \frac{d \sigma_z}{z - \zeta} = R(\zeta)
\]
where $R(\zeta)$ is rational.

 \item The left side is exactly the Cauchy transform of $\chi_D$, the characteristic function of $D$!

\end{itemize}

\end{frame}

%%%%%%%%%%%%%%%%%%%%%%%%%%%%%%%%%%%%%%%%%%%%%%%%%%%%%%%%%%%%%%%%%%%%%%%%%%%%%%%%

\section{Properties of Quadrature Domains}

\begin{frame}

Need to understand the Cauchy transform of $\chi_D$ $\ldots$

\begin{lemma}
Let $D \subset \mathbb C$ be such that
\[
\int_{D} \frac{d \sigma_z}{\vert z \vert} < \infty.
\]
Define
\[
S(\zeta) = \int_D \frac{d \sigma_z}{z - \zeta}.
\]
Then $S(\zeta)$ is continuous on $\mathbb C$, is $O(\vert \zeta \vert^{1/2})$ as $\zeta \rightarrow \infty$ and for $\zeta \in D$ satisfies
\[
S(\zeta) = - \pi \overline \zeta + g(\zeta)
\]
where $g$ is continuous on $\overline D$ and holomorphic on $D$.
\end{lemma}

\end{frame}
%%%%%%%%%%%%%%%%%%%%%%%%%%%%%%%%%%%%%%%%%%%%%%%%%%%%%%%%%%%%%%%%%%%%%%%%

\begin{frame}
 \frametitle{Proof}

 \begin{itemize}

 \item $S(\zeta) = \chi_D \ast 1/z$ and hence $S$ is continuous on $\mathbb C$.

 \item Take a test function $\phi$ and compute

\begin{align*}
\frac{\partial S}{\partial \overline \zeta} (\phi) &= - \int_{\mathbb C} \bigg( \frac{\partial \phi}{\partial \overline \zeta} \int_{\mathbb C} \frac{\chi_D(z)}{z - \zeta} \; d \sigma_z 
                                                                           \bigg) d \sigma_{\zeta}\\
                                                   &= - \int_{\mathbb C} \chi_D(z) \bigg( \int_{\mathbb C} \frac{1}{z - \zeta} \frac{\partial \phi}{\partial \overline \zeta} \; d 
                                                                            \sigma_{\zeta} \bigg) d \sigma_z.
\end{align*}

 \item The inner integral $ = \pi \phi(z)$ and hence $\frac{\partial S}{\partial \overline \zeta} = - \pi \chi_D(\zeta)$ as a distribution.

 \item $S(\zeta)$ is holomorphic for $\zeta \in \mathbb C \setminus D$ and in $D$, the function $g = S + \pi \overline \zeta$ satisfies 
${\partial g}/{\partial \overline \zeta} = 0.$


\end{itemize}

\end{frame}

%%%%%%%%%%%%%%%%%%%%%%%%%%%%%%%%%%%%%%%%%%%%%%%%%%%%%%%%%%%%%%%%%%%

\begin{frame}
 \frametitle{Proof Contd.}

 \begin{itemize}

 \item Let $D_1 = \{z : \vert z - \zeta \vert \le \vert \zeta \vert^{1/2}\}$ and $D_2 = \mathbb C \setminus D_1$.

\medskip

 \item Then
\[
\vert S(\zeta) \vert \le \int_{D_1} \frac{\vert \chi_D(z) \vert}{\vert z - \zeta \vert} \; d\sigma_z + \int_{D_2} \ast.
\]
 
\medskip

 \item The first integral is at most 
\[
\int_{\vert w \vert \le \vert \zeta \vert^{1/2}} \frac{d \sigma_w}{\vert w \vert} \lesssim \vert \zeta \vert^{1/2},
\]

\end{itemize}

\end{frame}
%%%%%%%%%%%%%%%%%%%%%%%%%%%%%%%%%%%%%%%%%%%%%%%%%%%%%%%%%%%%%%%%%

\begin{frame}

 \begin{itemize}

 \item while the second one can be written as
\[
\int_{D_2} \ast \le \int_{\mathbb C}  \frac{\vert z \vert}{\vert z -\zeta \vert} \cdot \frac{\vert \chi_D(z) \vert}{\vert z\vert} \; d \sigma_z
\]
where the term
\[
\frac{\vert z \vert}{\vert z -\zeta \vert} \le 1 + \frac{\vert \zeta \vert}{\vert z -\zeta \vert} \le 1 +  \vert \zeta \vert^{1/2}. 
\]
 
 \item Hence $S(\zeta) = O(\vert \zeta \vert^{1/2})$ for large $\zeta$.


 \end{itemize}

\end{frame}

%%%%%%%%%%%%%%%%%%%%%%%%%%%%%%%%%%%%%%%%%%%%%%%%%%%%%%%%%%%%%%%%%%%%

\begin{frame}
 \frametitle{The Aharonov--Shapiro Theorem}

The following are equivalent:

\begin{enumerate}

\item[(i)] $D$ is a quadrature domain.

\item[(ii)] There is a distribution $\alpha$ supported at finitely many points in $D$ (the nodes!) and which is a finite linear combination of Dirac masses and their derivatives at 
these points, such that

\[
\int _D f = \langle \alpha, f \rangle.
\]

\end{enumerate}

\end{frame}

%%%%%%%%%%%%%%%%%%%%%%%%%%%%%%%%%%%%%%%%%%%%%%%%%%%%%%%%%%%%%%%%%%%%

\begin{frame}

\begin{enumerate}

\item[(iii)] There is a rational function $R(\zeta)$ with all poles in $D$ such that $S(\zeta) = R(\zeta)$ for $\zeta \in \mathbb C \setminus D$.

\item[(iv)] There exists a meromorphic function in $D$, say $h$ with finitely many poles in $D$ such that $h$ is continuous on $\overline D$ and 
\[
h(z) = \overline z
\]
for  $z \in \partial D$.

\end{enumerate}

\medskip

{\it Remark:} Connection with the Schwarz reflection principle and the `analytic' Dirichlet problem.

\end{frame}

%%%%%%%%%%%%%%%%%%%%%%%%%%%%%%%%%%%%%%%%%%%%%%%%%%%%%%%%%%%%%%%%%%%

\begin{frame}
 \frametitle{Pointers to the Proof:}
\begin{itemize}

\item Suppose (ii) holds. Take $f(z) = 1/(z - \zeta)$ for $\zeta \in \mathbb C \setminus D$. Then
\[
S(\zeta) = \int_D \frac{d \sigma_z}{z - \zeta} = \langle \alpha, 1/(z - \zeta) \rangle
\]
which is a rational function, say $R(\zeta)$ with poles exactly at the nodes. Thus (iii) holds.

\item Conversely, if (iii) holds, then
\[
S(\zeta) = \int_D \frac{d \sigma_z}{z - \zeta} = R(\zeta) = p(\zeta)/q(\zeta).
\]

\end{itemize}

\end{frame}

%%%%%%%%%%%%%%%%%%%%%%%%%%%%%%%%%%%%%%%%%%%%%%%%%%%%%%%%%%%%%%

\begin{frame}

Since $S(\zeta) = O(\vert \zeta \vert^{1/2})$, $\deg p \le \deg q$, i.e., the partial fraction decomposition of $R$ does not have a holomorphic 
part. Thus, 
\[
R(\zeta) = \langle \alpha, 1/(z - \zeta) \rangle
\]
for an appropriate $\alpha$. The support of $\alpha$ is exactly the set of poles of $R(\zeta)$. Thus the quadrature identity holds for functions of the form $1/(z - \zeta)$, $\zeta \in 
\mathbb C \setminus D$. By Ahlfors--Bers, such functions are dense in our test class and hence (ii) holds.

\end{frame}

%%%%%%%%%%%%%%%%%%%%%%%%%%%%%%%%%%%%%%%%%%%%%%%%%%%%%%%%%%%%%%%%%%

\begin{frame}
 \frametitle{Why are (iii) and (iv) equivalent?}

Suppose (iii) holds, i.e., $S(\zeta) = R(\zeta)$ on $\mathbb C \setminus D$. Let
\[
h(\zeta) = \pi^{-1}(g(\zeta) - R(\zeta)). 
\]

\begin{itemize}

\item Then $h$ is meromorphic on $D$ and has finitely many poles in $D$.

\item $h$ extends continuously to $\partial D$ since $g$ does and on $\partial D$,
\[
h(\zeta) = \pi^{-1}(S(\zeta) + \pi \overline \zeta - R(\zeta)) = \overline \zeta.
\]

\end{itemize}

\end{frame}

%%%%%%%%%%%%%%%%%%%%%%%%%%%%%%%%%%%%%%%%%%%%%%%%%%%%%%%%%%%%%%%%%%

\begin{frame}
 \frametitle{Boundaries of Quadrature Domains}

\begin{theorem}
Let $D$ be a quadrature domain. Then there exists a nonconstant polynomial $P \in \mathbb R[X, Y]$, irreducible over $\mathbb C$ such that
\[
\partial D \subset \{z = x + iy : P(x, y) = 0\}.
\]
\end{theorem}

Assuming this --

\begin{itemize}
 
 \item Let
\[
F(z) =
\begin{cases}
  S(z), &\text{if $z \in \mathbb C \setminus D$;}\\
  g(z) - \pi h(z), &\text{if $z \in D$}.
\end{cases}
\]

 \item $F$ is meromorphic away from $\partial D$, continuous on $\mathbb C$ and is $O(\vert \zeta \vert^{1/2})$ far away. Hence it is rational!

\end{itemize}

\end{frame}

%%%%%%%%%%%%%%%%%%%%%%%%%%%%%%%%%%%%%%%%%%%%%%%%%%%%%%%%%%%%%%%%%

\begin{frame}
 \frametitle{Proof of the Theorem}

\begin{lemma}
Let $D \subset \mathbb C$ be bounded. Suppose $f, g \in \mathcal M(D)$ with only finitely many poles. Suppose further that both extend continuously to $\partial D$ and take real values 
there. Then there is a polynomial $P \in \mathbb R[X, Y]$, irreducible over $\mathbb C$ such that $P(f(z), g(z)) = 0$ on $D$.
\end{lemma}

\begin{itemize}
 \item $m = $ the no. of poles of $f, g$ with multiplicity and $\mathcal P = $ the set of poles of $f, g$.

 \item For $n \ge 1$, consider $\{ f^j g^k : j, k \ge 0, 1 \le j +k \le n\}$.

 \item Exactly $n(n + 3)/2$ such functions with poles in $\mathcal P$ and the total multiplicity is at most $mn$.

\end{itemize}

\end{frame}

%%%%%%%%%%%%%%%%%%%%%%%%%%%%%%%%%%%%%%%%%%%%%%%%%%%%%%%%%%%%%%%%%%%%

\begin{frame}

\begin{itemize}
 \item The vector space generated by the principal parts of $f^j g^k$ has real dimension $\le 2mn$.

\medskip

 \item Choose $n$ so that $n(n + 3)/2 > 2mn$!

\medskip

 \item Then there is $Q \in \mathbb R[X, Y]$ such that $Q(f, g) \in \mathcal O (D)$ and is real on $\partial D$. Then $Q(f, g) \equiv 0$ on $D$.

\medskip

 \item Choose such a $P$ with least degree $d$. Suppose that $P = P_1 P_2$ where $P_1, P_2 \in \mathbb C[X, Y]$ and have degrees $< d$.

\end{itemize}

\end{frame}
%%%%%%%%%%%%%%%%%%%%%%%%%%%%%%%%%%%%%%%%%%%%%%%%%%%%%%%%%%%%%%%

\begin{frame}

 \begin{itemize}

 \item Both $P_1(f, g), P_2(f, g)$ are meromorphic on $D$ and $P(f, g) = 0$ on $D$. Hence, say $P_1(f, g) \equiv 0$ on $D$.

\medskip

 \item Write $P_1 = P_3 + i P_4$ and note that 
\[
P_3(f, g) + i P_4(f, g) = 0
\]
on $\partial D$. Since $f, g$ are real on $\partial D$, it follows that $P_3(f, g), P_4(f, g) \equiv 0$ on $D$. 

\medskip

 \item But $\deg P_3, \deg P_4 < d$!

\end{itemize}

\end{frame}

%%%%%%%%%%%%%%%%%%%%%%%%%%%%%%%%%%%%%%%%%%%%%%%%%%%%%%%%%%%%%%%%%%%%%

\begin{frame}

\begin{itemize}

 \item Take $h$ meromorphic on $D$ with finitely many poles and $h(z) = \overline z$ on $\partial D$.

\medskip

 \item Let $f(z) = (z + h(z))/2$ and $g(z) = (z - h(z))/2$. Note that $f(z) = x, g(z) = y$ on $\partial D$.

\medskip

 \item There is a $P \in \mathbb R[X, Y]$ such that $P(f, g) = P(x, y) = 0$ on $\partial D$.

\end{itemize}

\end{frame}

%%%%%%%%%%%%%%%%%%%%%%%%%%%%%%%%%%%%%%%%%%%%%%%%%%%%%%%%%%%%%%%%%%%%%

\begin{frame}

 \frametitle{Remarks}
 
\begin{itemize}
  
 \item The region bounded by a triangle is not a quadrature domain.

 \item An annulus is not a quadrature domain.

 \item The region bounded by an ellipse is also not a quadrature domain. Use: If
\[
P = P_n + P_{n-1} + \ldots + P_0
\]
 is the decomposition into homogeneous terms, then $P_n$ is divisible by $x^2 + y^2$.

\end{itemize}

\end{frame}

%%%%%%%%%%%%%%%%%%%%%%%%%%%%%%%%%%%%%%%%%%%%%%%%%%%%%

\begin{frame}
 \frametitle{A characterization of the disc: Epstein--Schiffer}

Suppose that $0 \in D$, that $D$ is bounded and 
\[
\int_D f d \sigma_z = a f(0)
\]
for all $f$ in our test class. Then $h$ has a simple pole at $z = 0$ and $h(z) = \overline z$ on $\partial D$. So $z h(z) = z \overline z =$ constant on $\partial D$. This means that 
$D$ is a disc.


\end{frame}

%%%%%%%%%%%%%%%%%%%%%%%%%%%%%%%%%%%%%%%%%%%%%%%%%%%%%%%%%%%%%

\section{Constructing Quadrature Domains}

\begin{frame}
 \frametitle{Constructing Quadrature Domains}

\begin{theorem}

A bounded simply connected domain $D \subset \mathbb C$ is a quadrature domain if and only if its Riemann map 
\[
\phi : \Delta \rightarrow D
\]
is rational with all poles outside $\overline \Delta$. In fact, if $z_i$'s are the nodes in $D$ and $t_i = \phi^{-1}(z_i)$, then the poles of $\phi$ are exactly at $1/\overline t_i$.

\end{theorem}

\end{frame}

%%%%%%%%%%%%%%%%%%%%%%%%%%%%%%%%%%%%%%%%%%%%%%%%%%%%%%

\begin{frame}
 \frametitle{Proof}

\begin{itemize}
 
 \item $\phi$ extends continuously up to $\partial \Delta$.

 \item Then $h(\phi(t)) = \overline{\phi(t)}$ for $\vert t \vert = 1$. 

 \item Define 
\[
R(t) =
 \begin{cases}
  \overline{\phi(1/\overline{t})}, &\text{if $\vert t \vert > 1$;}\\ 
  h(\phi(t)),                      &\text{if $\vert t \vert \le 1$.}     
 \end{cases}
\]

 \item $R$ is rational. So is $\phi(t) = \overline{R(1/\overline{t})}$. 

\end{itemize}

\end{frame}

%%%%%%%%%%%%%%%%%%%%%%%%%%%%%%%%%%%%%%%%%%%%%%%%%%%%%%

\section{Examples}

\begin{frame}
 \frametitle{Illustrating this technique}

\begin{itemize}

\item Recall that $\psi(z) = (z + 1/z)/2$ conformally maps $\mathbb C \setminus \overline \Delta$ onto $\mathbb C \setminus [-1, 1]$.

\item For $R > 1$, the circle $C_R$ is mapped onto an ellipse $E_R$ where
\[
x = \frac{1}{2}\bigg(R + \frac{1}{R} \bigg) \cos \theta, y = \frac{1}{2} \bigg( R - \frac{1}{R} \bigg) \sin \theta.
\]

\item Thus $\psi(Rz)$ maps $\mathbb C \setminus \overline \Delta$ onto $\mathbb C \setminus \overline E_R$.

\end{itemize}

\end{frame}

%%%%%%%%%%%%%%%%%%%%%%%%%%%%%%%%%%%%%%%%%%%%%%%%%%%%%%%%%

\begin{frame}

\begin{itemize}

 \item Thus $z \mapsto 1/\overline{\psi (R/ \overline z)}$ is a conformal map from $\Delta$ onto the region bounded by inverting $E_R$ under the map $z \mapsto 1/\overline z$. 

 \item The boundary of this region, say $N_R$ is given by
\[
(x^2 + y^2)^2 - a^{-2} x^2 - b^{-2}y^2 = 0
\]
 where $a =  (R + 1/R)/2$ and $b = (R - 1/R)/2$.

 \item The conformal map on the disc is $\phi(z) = 2Rz/(R^2 + z^2)$ which has poles at $z = \pm i R$.


\end{itemize}

\end{frame}

%%%%%%%%%%%%%%%%%%%%%%%%%%%%%%%%%%%%%%%%%%%%%%%%%%%%%%%%%%%

\begin{frame}

\begin{itemize}

 \item Invert the poles to get $\pm i/R$ and note that $\phi(\pm i/R) = \pm 2i/(R^2 - R^{-2})$. These are the nodes in $N_R$!

 \item To get the quadrature identity, take $f$ holomorphic in a neighbourhood of $\overline N_R$ and note
\begin{align*}
\int_{N_R} f \; d \sigma_z &= (1/2i) \int_{\partial N_R} f(z) \overline z \;dz\\
                         &= (1/2i) \int_{\partial N_R} f(z) h(z) \;dz\\ 
                         &= 2\pi (a_1 f(z_1) + a_2f(z_2))  
\end{align*}
where $a_1, a_2$ are the residues of $h(z)$ at the simple poles $z_1, z_2$ above.

\end{itemize}

\end{frame}

%%%%%%%%%%%%%%%%%%%%%%%%%%%%%%%%%%%%%%%%%%%%%%%%%%%%%%%%%%%%%%%

\begin{frame}
 \frametitle{Ubiquity of quadrature domains: Gustafsson's Theorem}

\begin{theorem}
Let $D$ be a smoothly bounded domain with finite connectivity. Then there exists a quadrature domain $G$ as close as we want to $D$ in the $C^{\infty}$ topology.
\end{theorem}

{\it Remark:} An annulus is not a quadrature domain, but there are plenty of such domains nearby!

\end{frame}

%%%%%%%%%%%%%%%%%%%%%%%%%%%%%%%%%%%%%%%%%%%%%%%%%%%%%%%%%%%%%%%%%%%%%

\section{What happens in higher dimensions?}

\begin{frame}
 \frametitle{The story in higher dimensions}
\begin{center}
$\ldots$ is yet to take off.
\end{center}
\end{frame}

%%%%%%%%%%%%%%%%%%%%%%%%%%%%%%%%%%%%%%%%%%%%%%%%%%%%%%%%%%

\begin{frame}
 \frametitle{References}

\begin{itemize}

\item D. Aharonov, H. Shapiro: {\it Domains on which analytic functions satisfy quadrature identities}, J. Anal. Math. ${\bf 30}$, (1976), 39--73.

\item P. J. Davis: {\it The Schwarz function and its applications}, The Carus Math. Monographs, ${\bf 17}$, (1974).

\end{itemize}

\end{frame}

%%%%%%%%%%%%%%%%%%%%%%%%%%%%%%%%%%%%%%%
\begin{frame}

\begin{center}
 Thank You
\end{center}

\end{frame}

%%%%%%%%%%%%%%%%%%%%%%%%%%%%%%%%%%%%%%%%%%%%%%%%%%%%%%%%%%%


\end{document}

